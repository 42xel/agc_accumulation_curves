\newcommand{\FRIDGE}[1]{%
  {\sf\color{VeryLightBlue}{#1}}%	comment to hide
%  {\sf\color{Blue}{#1}}%			To make it more visible
}

%comment whole command to turn todo into error
\newcommand{\todo}[1]{%
{\sf\color{Red}{#1}}%
}




%%%%%%%%%%%%%%%%%%%%%%%%%% 
%% Writing         %%
%%%%%%%%%%%%%%%%%%%%%%%%%% 

% \newcommand{\Tild}{{\char126}}
% \newcommand{\Arob}{{\char64}}

\providecommand{\Latin}[1]{\text{#1}\xspace} %% Latin is not \it or \sl in English
\providecommand{\viceversa}{\Latin{vice-versa}}
\providecommand{\EtAl}{\Latin{{et al.}}}
\providecommand{\ie}{\Latin{i.e.}}
\providecommand{\eg}{\Latin{e.g.}}



%%%%%%%%%%%%%%%%%%%%%%%%%%%%%%%%%%%%%%%%%% 
%% Theorems & Environnements       %%
%%%%%%%%%%%%%%%%%%%%%%%%%%%%%%%%%%%%%%%%%% 

\newtheorem{theorem}{Theorem}
\newtheorem{corollary}[theorem]{Corollary}
\newtheorem{conjecture}[theorem]{Conjecture}
\newtheorem{proposition}[theorem]{Proposition}
\newtheorem{lemma}[theorem]{Lemma}

%{%
%	\theoremstyle{definition}%
%	\newtheorem{definition}[theorem]{Definition}
%}

\newtheorem{Theorem}{Theorem}
\newtheorem{Corollary}[Theorem]{Corollary}
\newtheorem{Conjecture}[Theorem]{Conjecture}
\newtheorem{Proposition}[Theorem]{Proposition}
\newtheorem{Lemma}[Theorem]{Lemma}

%{% \theorembodyfont{\rmfamily}
%	\theoremstyle{definition}%
%	\newtheorem{Definition}[Theorem]{Definition}
%}

\newenvironment{Proof}{\paragraphe{\it Proof} }{ {$q.e.d.$\smallskip}}
\newenvironment{Remark}{\paragraphe{Remark}}{\smallskip}

\newcommand{\RefFigure}[1]{Figure\,\ref{#1}}
\newcommand{\RefFig}[1]{Fig.\,\ref{#1}}

\newcommand{\RefTHeorem}[1]{Theorem\,\ref{#1}}
\newcommand{\RefTh}[1]{Th.\,\ref{#1}}
\newcommand{\RefLem}[1]{Lem.\,\ref{#1}}
\newcommand{\RefLemma}[1]{Lemma\,\ref{#1}}
\newcommand{\RefCor}[1]{Cor.\,\ref{#1}}
\newcommand{\RefCon}[1]{Conj.\,\ref{#1}}

\newcommand{\RefDef}[1]{Def.\,\ref{#1}}

\newcommand{\RefPRoposition}[1]{Proposition\,\ref{#1}}
\newcommand{\RefProp}[1]{Prop.\,\ref{#1}}

\newcommand{\RefEq}[1]{(\ref{#1})}

\newcommand{\RefSection}[1]{Section\,\ref{#1}}
\newcommand{\RefSec}[1]{Sect.\,\ref{#1}}

\newcommand{\RefSubsection}[1]{Subsection\,\ref{#1}}
\newcommand{\RefSubsec}[1]{Subsect.\,\ref{#1}}

\newcommand{\RefAlgorithm}[1]{Algorithm\,\ref{#1}}
\newcommand{\RefAlgo}[1]{Algo.\,\ref{#1}}



\newcommand{\VCenter}[1]{\raisebox{-.45\height}{\raisebox{\depth}{#1}}}


\newcounter{Example}
\newcommand{\PicExample}[2]{%
	\begin{figure}[hbt]
		\centering
		\mbox{}\hfill%%
		\includegraphics[scale=#2]{test_#1.pdf}%
		\hfill%
		\hfill%
		% \includegraphics[scale=#2]{test_#1_sim.pdf}%    
		\hfill\hfill\mbox{}%
		\stepcounter{Example}
		\caption{Example \theExample.}
		\label{fig:annexe:example:#1}
	\end{figure}
}



\newenvironment{MSlist}[1][]{%
	\begingroup\small\footnotesize\scriptsize%
	\begin{tabular}{@{}r@{\,}c@{\,}>{\begin{math}}c<{\end{math}}@{}}%
		&\,{\small\footnotesize\bf #1Meta-signal\,}%
		&\,\text{\small\footnotesize\bf Speed}\!%
		\\[.1em]
	}{%
	\end{tabular}%
	\endgroup%
}

\newenvironment{CRlist}{%
	\begingroup\small\footnotesize\scriptsize%
	\begin{tabular}{@{}r@{\,}>{\raggedleft\arraybackslash\{\,}r<{\,\}}@{\,$\to$\,\{\,}>{\raggedright\arraybackslash}l<{\,\}}}%
	}{%
	\end{tabular}\endgroup}

\newenvironment{AugCRlist}{%
	\begingroup\small\footnotesize\scriptsize%
	\begin{tabular}{@{}r@{\,}>{\raggedleft\arraybackslash\{\,}r<{\,\}}@{\,}%
			r%
			@{\,\{\,}>{\raggedright\arraybackslash}l<{\,\}}}%
	}{%
	\end{tabular}\endgroup}





%%%%%%%%%%%%%%%%%%%%%%%%%%%%%% 
%% FOR V-spacing
%%%%%%%%%%%%%%%%%%%%%%%%%%%%%% 

\newcommand{\NTVS}{\vspace{-1mm}}
\newcommand{\NVS}{\vspace{-2mm}}



%%%%%%%%%%%%%%%%%%%%%%%%%%%%%% 
%% FOR setting the unit length at once for picture, pstricks and tikz at once.
%%%%%%%%%%%%%%%%%%%%%%%%%%%%%% 

\providecommand{\SetUnitlength}[1]{%
	\setlength{\unitlength}{#1}%
	\ifx\tikzpicture\undefined\relax\else%
	\tikzset{x=#1}%
	\tikzset{y=#1}%
	\fi%
	\ifx\PSTricksLoaded\undefined\relax\else%
	\psset{unit=#1}%
	\fi%
}


%%%%%%%%%%%%%%%%%%%%%%%%%%%%%% 
%% TIKZ style for ray
%%%%%%%%%%%%%%%%%%%%%%%%%%%%%% 

\tikzstyle{ray_style}=[fill=yellow!60!white,draw=none]

%%%%%%%%%%%%%%%%%%%%%%%%%%%%%% 
%% For simplifying MS table constructions 
%%%%%%%%%%%%%%%%%%%%%%%%%%%%%% 

\newcommand{\foreachASig}[1]{%
	&\csname Sig#1\endcsname &0\\
	&\csname SigSL#1\endcsname &-1\\
	&\csname SigSR#1\endcsname &1
}

% \newcommand{\foreachASigParam}[3]{%
%   &\csname Sig#1\endcsname &#2\\
%   &\csname SigSL#1\endcsname &#3\\
%   &\csname SigSR#1\endcsname &#2
% }



%%%%%%%%%%%%%%%%%%%%%%%%%%%%%% 
%%%%%%%%%%%%%%%%%%%%%%%%%%%%%% 
% ;;; Local Variables: ***
% ;;; eval: (flyspell-mode) ***
% ;;; eval: (ispell-change-dictionary "american") ***
% ;;; eval: (flyspell-buffer) ***
% ;;; End: ***
