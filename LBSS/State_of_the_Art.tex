\section{State of the Art}
\subsection{BSS}

The "traditional" BSS model and small variants, as described by Blum, Shub and Smale in \cite{blum1989}.

\subsubsection{basic}

\begin{itemize}
	\item $\mathcal{I}$: input space, typically $\mathbb{R}^n$
	\item $\mathcal{O}$: output space, typically $\mathbb{R}^n$
	\item $\mathcal{S}$: computation space, typically $\mathbb{R}^n$
	\item a finite oriented graph (corresponding to states of a Turing machine or lines of a program)
\end{itemize}

The graph has four types of nodes. Nodes are designed by the letter $\MachineNode$:

\begin{itemize}
	\item A single starting node, called input node, written $1$.
	It has a single next node $\gamma_1$ and no antecedent and has an associated map $ I : \mathcal{I} \mapsto \mathcal{S}$
	\item Computation nodes, with a single next node $\gamma_{\MachineNode}$ and the associated computational map, $g_{\MachineNode} \mathcal{S} \mapsto \mathcal{S}: $, polynomial/rational depending.
	\item Branching node, with a non zero polynomial branching function $h_{\MachineNode} : \mathcal{S} \mapsto \mathcal{R}$, and two next nodes : $\gamma^+_{\MachineNode}$ and $\gamma^-_{\MachineNode}$.
	\item Output node, with no next node and an associated map $ O_{\MachineNode} : \mathcal{S} \mapsto \mathcal{O}$
\end{itemize}

The computation is as expected: we are given an input $i\in\mathcal{I}$, the first node instructs us how to inject it in $\mathcal{S}$, giving us a value $x$, and we proceed to the next node.
A computation node instructs us to update $x$ : $x := g_{\MachineNode}(x)$, and proceed to the next node.
A branching node tells us to go to the node $\gamma^-_{\MachineNode}$ if $h_{\MachineNode}(x) < 0$ and to $\gamma^+_{\MachineNode}$ otherwise.
An output nodes tells us to stop and let us know how to interpret the result.


\emph{Remark:} The graph approach is slightly different from an instruction flow with line numbers in that a node can have more than one non branching antecedent.
The graph approach seems more natural for signal machine as well: nodes are canonically handled with meta signals, and transition with collision rules.

\subsubsection{Infinite dimension}
Input, computation and output space may be of infinite dimension.
Polynomial/rational functions are defined in such a way they only act upon a finite number of dimensions.
A fifth kind of node allows to copy the $i$-th component into the $j$-th.
$i$ and $j$ are integrated into the computation space so as to be modified by computation nodes, with limitation so that they stay integer.

\subsubsection{\LBSS}
Linear BSS is the same as BSS, but with linear functions/map instead of rational/polynomial.

\subsection{Linear $\mathbb{R}$ - URM}
Linear $\mathbb{R}$ - URM was first introduced by Durand Lose in \cite{durand-lose07cie}.
It is equivalent to \LBSS, but simpler to simulate.
From a user perspective, it has pros, like more powerful addressing, and cons like very atomic computation (at least one line of code per elementary operation).

\begin{itemize}
	\item $(A_i)_{i\in I}$ addresses, $I$ finite constant.
	\item $(R_j)_{j\in J}$ registers, $J$ finite extendable.
	\item $(R_(i))_{i\in I}$ way to address the registers indirectly.
	\item $X$ accumulator.
\end{itemize}

instructions: 
\begin{itemize}
	\item addressing $\text{inc}A_i$, $\text{dec}A_i$, if $0<A_i \text{ goto } n$ for the registers.
	\item $\text{load} R_{i}$, $\text{store} R_{i}$, $\text{add} R_{i}$
	\item $\text{mul}\ConstantMul$
	\item if $0<X$ goto $n$ for the registers.
\end{itemize}

%\subsection{Linear $\mathbb{R}$ - LRM}
%Same as above, but with a finite number of registers. Easiest to simulate, but a priori not equivalent, so addressing must stay in the back of my mind even if I forego them for now.
%
%\begin{itemize}
%	\item $(R_j)_{j\in J}$ registers, $J$ finite constant.
%	\item $X$ accumulator.
%\end{itemize}
%
%instructions: 
%\begin{itemize}
%	\item addressing $\text{inc}A_i$, $\text{dec}A_i$, if $0<A_i \text{ goto } n$ for the registers.
%	\item $\text{load} R_{i}$, $\text{store} R_{i}$, $\text{add} R_{i}$
%	\item $\text{mul}\ConstantMul$
%	\item if $0<X$ goto $n$ for the registers.
%\end{itemize}