\section{Unary binary tree}

The idea is to build a unary binary tree, with an LBSS machine providing the orders, by stopping on a given terminal node.

\subsection{Primitives}
\subsubsection{Delay}
\TODO{txt for illustration}
Doing nothing during a given amount of time carefully measured with auxiliary bouncing signal.

\begin{figure}[hbt]
	\centering
	\small%
	\SetUnitlength{.14\textwidth}%
	\includegraphics[height=\unitlength]{ubt_upright_to_Delay_.pdf}%
	\caption{Delay from a upward right branch.}
	\label{fig:delay:upright}
\end{figure}

\begin{figure}[hbt]
	\centering
	\small%
	\SetUnitlength{.14\textwidth}%
	\includegraphics[height=\unitlength]{ubt_right_to_Delay_.pdf}%
	\caption{Delay from a rightward branch.}
	\label{fig:delay:right}
\end{figure}


\subsubsection{split}
\TODO{txt for illustration}
The information is duplicated and split into a right and a left branch.

\begin{figure}[hbt]
	\centering
	\small%
	\SetUnitlength{.14\textwidth}%
	\includegraphics[height=\unitlength]{ubt_right_to_Split_.pdf}%
	\caption{Split from a rightward branch.}
	\label{fig:split:right}
\end{figure}

\begin{figure}[hbt]
\centering
\small%
\SetUnitlength{.14\textwidth}%
\includegraphics[height=\unitlength]{ubt_upleft_to_Split_.pdf}%
\caption{Splitting from a upward left branch.}
\label{fig:split:upleft}
\end{figure}


\subsection{Wiring with LBSS}

There is one LBSS machine coding a class of parameterized functions.
The initial configuration corresponds to the parameters.
Two new kinds of states are added:

\begin{itemize}
	\item \Delay: signals the next tree node has a single child and suspends computation until the next tree node.
	It has exactly one next state, indicating where to resume computation.
	\item \Split: signals the next tree node has two children (and it halves the scale) and suspends computation until the next tree node.
	It has exactly two next states, indicating where to resume computation on each children.
	In the code, it translates to 2 mandatory labels, corresponding to left and right, in that order.
\end{itemize}

Example code for a delay:
\begin{minted}{C}
	Delay;
\end{minted}

Example code for a delay with a goto:
\begin{minted}{C}
	Delay; goto labelFromDelay
\end{minted}

Example code for a split:
\begin{minted}{C}
	Split labelFromSplitToLeft, labelFromSplitToRight
\end{minted}

Example code for slanted:
\begin{minted}[breaklines, linenos]{C}
left:
Split left, right;
right:
Delay left;
end;
machine_ubtTest_rec
\end{minted}
This machine will recursively draw a tree where leftward and upward branches are followed by a split and right branches are followed by a delay.

\subsection{AGC Implementation}
\TODO{fluff up (mscr) and tidy up, cite sfss}
The tree is organized with Macro Signals: a band of signals bounded by the tree signal, and a bounding signal.

The tree steps are performed with the help of a pair bouncing signals
They correspond to both bounds of the macro signal.
The first intersection with bouncing signals starts the macro collision and the second marks the exact position of the Tree Node.

The information about whether the next node is $\Delay$ or $\Split$ is on the tree in an up branch, on the bound otherwise.


\begin{figure}[hbt]
	\centering
	\small%
	\SetUnitlength{1.0\textwidth}%
	\includegraphics[height=\unitlength]{ubt_agc.pdf}%
	\caption{Delay from a rightward branch.}
	\label{fig:ubtagc:example1}
\end{figure}

\subsubsection{Stopping Condition}
For now, it is assumed that the number of steps by the lbss machine is bounded, with a known upper bound.
\FRIDGE{(fridge) We assume it always end and we can compute it in finite time.
If it cannot, it fails (the simulation does not go further).}

If the machine ends on a split or a delay, the information is reported where  it should for the next node.
If it ends on an ending node, no information is reported and it prompts the stopping of this node.
Additionally -and optionally, a maximum depth, tracked by the tree meta-signals, is also used to stop the drawing of the tree.
It works by simply not prompting the computation once the maximum depth is reached, so that no result is reported.

\TODO{rerouting}
\TODO{lbss incrustration}