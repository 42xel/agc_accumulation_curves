
\documentclass[12pt,a4paper]{article}

%\usepackage{xspace}	%

\usepackage{amsmath}%% For \text
\usepackage{amssymb}  %% For various symbols and \mathbb

\usepackage{tikz}  %% for all graphics

\usepackage{minted}%% Display Python code


\usepackage{graphicx}
\graphicspath{{./Pictures/}}
\DeclareGraphicsExtensions{.jpeg,.jpg,.eps,.pdf}


\usepackage{subcaption}

\usepackage{tabularx}% For tabular

%\usepackage{enumitem}%% For spaces in liste, item… 

%\usepackage{subcaption} %%subcatption, subcaption box

%\renewcommand{\listingscaption}{Algorithm}


%%%%%%%%%%%%%%%%%%%%%%%%%%%%%% 
% These are JDL personnal packages
\usepackage{h_vocabulary} % For a correct use of Vocabulary, var definition…
% \usepackage[track]{h_vocabulary} % For a correct use of Vocabulary, var definition… make them appear in color

\usepackage{h_graphics} % for SetUnitlength and such
\usepackage{h_agc} % Some definitions are commands
% \usepackage[MSinTablesymbol]{h_agc} % Some definitions are commands Sig names appear in table of symbols


\usepackage{Add} %% Local .sty

\newcommand{\FRIDGE}[1]{%
  {\sf\color{VeryLightBlue}{#1}}%	comment to hide
%  {\sf\color{Blue}{#1}}%			To make it more visible
}

%comment whole command to turn todo into error
\newcommand{\todo}[1]{%
{\sf\color{Red}{#1}}%
}




%%%%%%%%%%%%%%%%%%%%%%%%%% 
%% Writing         %%
%%%%%%%%%%%%%%%%%%%%%%%%%% 

% \newcommand{\Tild}{{\char126}}
% \newcommand{\Arob}{{\char64}}

\providecommand{\Latin}[1]{\text{#1}\xspace} %% Latin is not \it or \sl in English
\providecommand{\viceversa}{\Latin{vice-versa}}
\providecommand{\EtAl}{\Latin{{et al.}}}
\providecommand{\ie}{\Latin{i.e.}}
\providecommand{\eg}{\Latin{e.g.}}



%%%%%%%%%%%%%%%%%%%%%%%%%%%%%%%%%%%%%%%%%% 
%% Theorems & Environnements       %%
%%%%%%%%%%%%%%%%%%%%%%%%%%%%%%%%%%%%%%%%%% 

\newtheorem{theorem}{Theorem}
\newtheorem{corollary}[theorem]{Corollary}
\newtheorem{conjecture}[theorem]{Conjecture}
\newtheorem{proposition}[theorem]{Proposition}
\newtheorem{lemma}[theorem]{Lemma}

%{%
%	\theoremstyle{definition}%
%	\newtheorem{definition}[theorem]{Definition}
%}

\newtheorem{Theorem}{Theorem}
\newtheorem{Corollary}[Theorem]{Corollary}
\newtheorem{Conjecture}[Theorem]{Conjecture}
\newtheorem{Proposition}[Theorem]{Proposition}
\newtheorem{Lemma}[Theorem]{Lemma}

%{% \theorembodyfont{\rmfamily}
%	\theoremstyle{definition}%
%	\newtheorem{Definition}[Theorem]{Definition}
%}

\newenvironment{Proof}{\paragraphe{\it Proof} }{ {$q.e.d.$\smallskip}}
\newenvironment{Remark}{\paragraphe{Remark}}{\smallskip}

\newcommand{\RefFigure}[1]{Figure\,\ref{#1}}
\newcommand{\RefFig}[1]{Fig.\,\ref{#1}}

\newcommand{\RefTHeorem}[1]{Theorem\,\ref{#1}}
\newcommand{\RefTh}[1]{Th.\,\ref{#1}}
\newcommand{\RefLem}[1]{Lem.\,\ref{#1}}
\newcommand{\RefLemma}[1]{Lemma\,\ref{#1}}
\newcommand{\RefCor}[1]{Cor.\,\ref{#1}}
\newcommand{\RefCon}[1]{Conj.\,\ref{#1}}

\newcommand{\RefDef}[1]{Def.\,\ref{#1}}

\newcommand{\RefPRoposition}[1]{Proposition\,\ref{#1}}
\newcommand{\RefProp}[1]{Prop.\,\ref{#1}}

\newcommand{\RefEq}[1]{(\ref{#1})}

\newcommand{\RefSection}[1]{Section\,\ref{#1}}
\newcommand{\RefSec}[1]{Sect.\,\ref{#1}}

\newcommand{\RefSubsection}[1]{Subsection\,\ref{#1}}
\newcommand{\RefSubsec}[1]{Subsect.\,\ref{#1}}

\newcommand{\RefAlgorithm}[1]{Algorithm\,\ref{#1}}
\newcommand{\RefAlgo}[1]{Algo.\,\ref{#1}}



\newcommand{\VCenter}[1]{\raisebox{-.45\height}{\raisebox{\depth}{#1}}}


\newcounter{Example}
\newcommand{\PicExample}[2]{%
	\begin{figure}[hbt]
		\centering
		\mbox{}\hfill%%
		\includegraphics[scale=#2]{test_#1.pdf}%
		\hfill%
		\hfill%
		% \includegraphics[scale=#2]{test_#1_sim.pdf}%    
		\hfill\hfill\mbox{}%
		\stepcounter{Example}
		\caption{Example \theExample.}
		\label{fig:annexe:example:#1}
	\end{figure}
}



\newenvironment{MSlist}[1][]{%
	\begingroup\small\footnotesize\scriptsize%
	\begin{tabular}{@{}r@{\,}c@{\,}>{\begin{math}}c<{\end{math}}@{}}%
		&\,{\small\footnotesize\bf #1Meta-signal\,}%
		&\,\text{\small\footnotesize\bf Speed}\!%
		\\[.1em]
	}{%
	\end{tabular}%
	\endgroup%
}

\newenvironment{CRlist}{%
	\begingroup\small\footnotesize\scriptsize%
	\begin{tabular}{@{}r@{\,}>{\raggedleft\arraybackslash\{\,}r<{\,\}}@{\,$\to$\,\{\,}>{\raggedright\arraybackslash}l<{\,\}}}%
	}{%
	\end{tabular}\endgroup}

\newenvironment{AugCRlist}{%
	\begingroup\small\footnotesize\scriptsize%
	\begin{tabular}{@{}r@{\,}>{\raggedleft\arraybackslash\{\,}r<{\,\}}@{\,}%
			r%
			@{\,\{\,}>{\raggedright\arraybackslash}l<{\,\}}}%
	}{%
	\end{tabular}\endgroup}





%%%%%%%%%%%%%%%%%%%%%%%%%%%%%% 
%% FOR V-spacing
%%%%%%%%%%%%%%%%%%%%%%%%%%%%%% 

\newcommand{\NTVS}{\vspace{-1mm}}
\newcommand{\NVS}{\vspace{-2mm}}



%%%%%%%%%%%%%%%%%%%%%%%%%%%%%% 
%% FOR setting the unit length at once for picture, pstricks and tikz at once.
%%%%%%%%%%%%%%%%%%%%%%%%%%%%%% 

\providecommand{\SetUnitlength}[1]{%
	\setlength{\unitlength}{#1}%
	\ifx\tikzpicture\undefined\relax\else%
	\tikzset{x=#1}%
	\tikzset{y=#1}%
	\fi%
	\ifx\PSTricksLoaded\undefined\relax\else%
	\psset{unit=#1}%
	\fi%
}


%%%%%%%%%%%%%%%%%%%%%%%%%%%%%% 
%% TIKZ style for ray
%%%%%%%%%%%%%%%%%%%%%%%%%%%%%% 

\tikzstyle{ray_style}=[fill=yellow!60!white,draw=none]

%%%%%%%%%%%%%%%%%%%%%%%%%%%%%% 
%% For simplifying MS table constructions 
%%%%%%%%%%%%%%%%%%%%%%%%%%%%%% 

\newcommand{\foreachASig}[1]{%
	&\csname Sig#1\endcsname &0\\
	&\csname SigSL#1\endcsname &-1\\
	&\csname SigSR#1\endcsname &1
}

% \newcommand{\foreachASigParam}[3]{%
%   &\csname Sig#1\endcsname &#2\\
%   &\csname SigSL#1\endcsname &#3\\
%   &\csname SigSR#1\endcsname &#2
% }



%%%%%%%%%%%%%%%%%%%%%%%%%%%%%% 
%%%%%%%%%%%%%%%%%%%%%%%%%%%%%% 
% ;;; Local Variables: ***
% ;;; eval: (flyspell-mode) ***
% ;;; eval: (ispell-change-dictionary "american") ***
% ;;; eval: (flyspell-buffer) ***
% ;;; End: ***
 %% Local newcommand

\JDLvocabulary{\NaturalSet}{\mathBBxspace{N}}{}{Natural set}
\JDLvocabulary{\RelativeSet}{\mathbb{Z}}{}{Relative set}%{\mathBBxspace{Z}}{}{Relative set}
\JDLvocabulary{\RationalSet}{\mathBBxspace{Q}}{}{Rational set}
\JDLvocabulary{\RealSet}{\mathcal{R}}{}{Real set}%{\mathBBxspace{R}}{}{Real set}

\JDLvocabulary{\functionCreateRLTMmachine}{function\_create\_lbss\_machine}{}{a pseudo function to create a machine}
\JDLvocabulary{\functionCreateRLTMconfiguration}{create\_lbss\_configuration}{}{a pseudo function to create a configuration}
\JDLvocabulary{\functionToAGCmachine}{toAGC\_machine}{}{a pseudo function to export a machine}
\JDLvocabulary{\functionToAGCconfig}{toAGC\_config}{}{a pseudo function to export a configuration}
\JDLvocabulary{\functionToAGCrun}{toAGC\_run}{}{a pseudo function to export a run}

\JDLvocabulary{\MachineNode}{\eta}{}{a node}
\JDLvocabulary{\NextNode}{\gamma_\MachineNode}{}{the next node}
\JDLvocabulary{\NextNodePile}{\NextNode'}{}{the next node}
\JDLvocabulary{\NextNodePlus}{\NextNode^+}{}{a next node}
\JDLvocabulary{\NextNodeEqual}{\NextNode^=}{}{a next node}
\JDLvocabulary{\NextNodeMinus}{\NextNode^-}{}{a next node}

\JDLvocabulary{\Tape}{x}{}{the tape}
\JDLvocabulary{\TapeHead}{p}{}{the position of the head of the tape}
%\JDLvocabulary{\HeadValue}{\Tape_\TapeHead}{}{the value at the head of the tape}
\JDLvocabulary{\HeadValue}{h}{}{the value at the head of the tape}
\JDLvocabulary{\AccValue}{a}{}{the value of the accumulator}
\JDLvocabulary{\Unit}{u}{}{the unit, a constant of value $1$ accessible at all times}

\JDLvocabulary{\ConstantAdd}{\beta}{}{an additive constant}
\JDLvocabulary{\ConstantMul}{\alpha}{}{a multiplicative constant}

\JDLvocabulary{\ConstantMulMax}{\alpha_{max}}{}{the biggest multiplicative constant}


\JDLvocabulary{\LBSS}{LBSS}{}{initialism for Linear Blum Shub Smale}
\JDLvocabulary{\RLTM}{LBSS}{}{initialism of the language name}
\JDLvocabulary{\extensionRLTM}{.lbss}{}{the lbss extension}

\JDLvocabulary{\goMS}{>>}{}{the go metasignal}

\JDLvocabulary{\returnMS}{<<}{}{the default return metasignal}
\JDLvocabulary{\returnMSZzero}{\returnMS_0}{}{the bottom get return metasignal}


\JDLvocabulary{\Split}{Split}{}{the split command}
\JDLvocabulary{\Delay}{Delay}{}{the delay command} %% Local vocabulary


\begin{document}


% Todo, understand the following comments and alter them to suit me

%% Elsevier article
% \begin{frontmatter}
% %   Title, authors and addresses
% %   use the thanksref command within \title, \author or \address for footnotes;
% %   use the corauthref command within \author for corresponding author footnotes;
% %   use the ead command for the email address,
% %   and the form \ead[url] for the home page:
% %   \title{Title\thanksref{label1}}
% %   \thanks[label1]{}
% %   \author{Name\corauthref{cor1}\thanksref{label2}}
% %   \ead{email address}
% %   \ead[url]{home page}
% %   \author{J{\'e}r{\^o}me \textsc{Durand-Lose}\thanksref{label1}}%\corauthref{cor1}\thanksref{label2}}
% %   \ead{jerome.durand-lose@univ-orleans.fr}
% %   \ead[url]{http\string://www.univ-orleans.fr/lifo/Members/Jerome.Durand-Lose}
% %   \corauth[cor1]{}
% %   \address{Address\thanksref{label3}}
% %   \thanks[label3]{}
% %   use optional labels to link authors explicitly to addresses:
% %   \author[label1,label2]{}
% %   \address[label1]{}
% %   \address[label2]{}
%   \begin{abstract}
% %     Text of abstract
%   \end{abstract}
%   \begin{keyword}
% %     keywords here, in the form: keyword \sep keyword
% %     PACS codes here, in the form: \PACS code \sep code
%   \end{keyword}
% \end{frontmatter}


\title{}
\author{%J{\'e}r{\^o}me \textsc{Durand-Lose}\thanks{%
		%    \texttt{http\string://www.univ-orleans.fr/lifo/Members/Jerome.Durand-Lose},
%		\texttt{jerome.durand-lose@univ-orleans.fr}}
%	\and  %
Aurélien \textsc{Emmanuel}
}

\date{\today{} (or version number)}



\maketitle

\begin{center}
	\small Université d'Orléans, INSA Centre Val de Loire,
	LIFO EA 4022,
	\\
	FR-45067 Orléans, France  
\end{center}


\begin{abstract}
	\parindent .75cm
	\TODO{write abstract}
\end{abstract}

{\renewcommand{\abstractname}{Key-Words}%
	
	\begin{abstract}
		\TODO{finish kwords}
		Abstract Geometrical Computation~;
		Cellular Automata~;
		Fractal~;
		Signal Machines.
	\end{abstract}%  
}

% ;;; Local Variables: ***
% ;;; eval: (flyspell-mode) ***
% ;;; eval: (ispell-change-dictionary "british" nil) ***
% ;;; eval: (flyspell-buffer) ***
% ;;; End: ***


\section{Intro}

This doc describe various models related to computation in discrete steps on real variables.
Then it expends on one that I'll use for AGC.


\section{Computation in a Ring}
\TODO{frame my goal in the bigger picture of ring computation}

From now on, $\RealSet$ is the set of real numbers, with it's usual ring structure and order.

\FRIDGE{$\mathcal{R}$ is a commutative ring with a unit and an order, typically $\mathbb{Z}$, $\mathbb{Q}$ or $\mathbb{R}$}



\section{State of the Art}
\subsection{BSS}

The "traditional" BSS model and small variants, as described by Blum, Shub and Smale in \cite{blum1989}.

\subsubsection{basic}

\begin{itemize}
	\item $\mathcal{I}$: input space, typically $\mathbb{R}^n$
	\item $\mathcal{O}$: output space, typically $\mathbb{R}^n$
	\item $\mathcal{S}$: computation space, typically $\mathbb{R}^n$
	\item a finite oriented graph (corresponding to states of a Turing machine or lines of a program)
\end{itemize}

The graph has four types of nodes. Nodes are designed by the letter $\MachineNode$:

\begin{itemize}
	\item A single starting node, called input node, written $1$.
	It has a single next node $\gamma_1$ and no antecedent and has an associated map $ I : \mathcal{I} \mapsto \mathcal{S}$
	\item Computation nodes, with a single next node $\gamma_{\MachineNode}$ and the associated computational map, $g_{\MachineNode} \mathcal{S} \mapsto \mathcal{S}: $, polynomial/rational depending.
	\item Branching node, with a non zero polynomial branching function $h_{\MachineNode} : \mathcal{S} \mapsto \mathcal{R}$, and two next nodes : $\gamma^+_{\MachineNode}$ and $\gamma^-_{\MachineNode}$.
	\item Output node, with no next node and an associated map $ O_{\MachineNode} : \mathcal{S} \mapsto \mathcal{O}$
\end{itemize}

The computation is as expected: we are given an input $i\in\mathcal{I}$, the first node instructs us how to inject it in $\mathcal{S}$, giving us a value $x$, and we proceed to the next node.
A computation node instructs us to update $x$ : $x := g_{\MachineNode}(x)$, and proceed to the next node.
A branching node tells us to go to the node $\gamma^-_{\MachineNode}$ if $h_{\MachineNode}(x) < 0$ and to $\gamma^+_{\MachineNode}$ otherwise.
An output nodes tells us to stop and let us know how to interpret the result.


\emph{Remark:} The graph approach is slightly different from an instruction flow with line numbers in that a node can have more than one non branching antecedent.
The graph approach seems more natural for signal machine as well: nodes are canonically handled with meta signals, and transition with collision rules.

\subsubsection{Infinite dimension}
Input, computation and output space may be of infinite dimension.
Polynomial/rational functions are defined in such a way they only act upon a finite number of dimensions.
A fifth kind of node allows to copy the $i$-th component into the $j$-th.
$i$ and $j$ are integrated into the computation space so as to be modified by computation nodes, with limitation so that they stay integer.

\subsubsection{\LBSS}
Linear BSS is the same as BSS, but with linear functions/map instead of rational/polynomial.

\subsection{Linear $\mathbb{R}$ - URM}
Linear $\mathbb{R}$ - URM was first introduced by Durand Lose in \cite{durand-lose07cie}.
It is equivalent to \LBSS, but simpler to simulate.
From a user perspective, it has pros, like more powerful addressing, and cons like very atomic computation (at least one line of code per elementary operation).

\begin{itemize}
	\item $(A_i)_{i\in I}$ addresses, $I$ finite constant.
	\item $(R_j)_{j\in J}$ registers, $J$ finite extendable.
	\item $(R_(i))_{i\in I}$ way to address the registers indirectly.
	\item $X$ accumulator.
\end{itemize}

instructions: 
\begin{itemize}
	\item addressing $\text{inc}A_i$, $\text{dec}A_i$, if $0<A_i \text{ goto } n$ for the registers.
	\item $\text{load} R_{i}$, $\text{store} R_{i}$, $\text{add} R_{i}$
	\item $\text{mul}\ConstantMul$
	\item if $0<X$ goto $n$ for the registers.
\end{itemize}

%\subsection{Linear $\mathbb{R}$ - LRM}
%Same as above, but with a finite number of registers. Easiest to simulate, but a priori not equivalent, so addressing must stay in the back of my mind even if I forego them for now.
%
%\begin{itemize}
%	\item $(R_j)_{j\in J}$ registers, $J$ finite constant.
%	\item $X$ accumulator.
%\end{itemize}
%
%instructions: 
%\begin{itemize}
%	\item addressing $\text{inc}A_i$, $\text{dec}A_i$, if $0<A_i \text{ goto } n$ for the registers.
%	\item $\text{load} R_{i}$, $\text{store} R_{i}$, $\text{add} R_{i}$
%	\item $\text{mul}\ConstantMul$
%	\item if $0<X$ goto $n$ for the registers.
%\end{itemize}

\section{Real Linear Turing Machine}
\TODO{Officialize rltm}

\subsection{theoretical definition}
\label{subsec:RLTMdef}
Much like Turing machine with reading and writing replaced by elementary operations enabling real linear algebra.

A machine is:
\begin{itemize}
	\item A bi-infinite tape made of cells, each capable of holding a number (an element of $\RealSet$).
	The tape is formally an element of $\RealSet^{\RelativeSet}$, $(x_i)_{i\in \RelativeSet}$
	\item A head pointing at somewhere on the tape.
	Formally, it's the index of a cell, and it is called $\TapeHead$.
	\item An accumulator, a number $\AccValue$ in $\RealSet$.
	This number is used for intermediate computation.
	\item A finite oriented graph, whose vertices are called the states of the machine.
	%\item A stack of node, analogous to a call stack. Formally, it is a finite list of nodes.
\end{itemize}

States (nodes of the graph) can be of the following type:

\begin{itemize}
	\item computation state,  with a single next state $\NextNode$ and an \emph{elementary computation} $g_\MachineNode$.
	\item Moving state, with an associated direction, \emph{left} or \emph{right}. It has a single next state $\NextNode$.
	\item Branching state, with three next states : $\NextNodePlus$, $\NextNodeEqual$ and $\NextNodeMinus$.
	%\emph A piling state, with two next states: $\NextNode$ and $\NextNodePile$.
	%\emph{remark:} Not necessary but too useful to pass on (allow to reuse states, %reducing their numbers, and natural implement in AGC).
	\item Terminal state, with no next state.
\end{itemize}
In addition, one state is marked as the starting state.


An elementary computation is made of the following:
\begin{itemize}
	\item a recipient: the accumulator ($\AccValue$) or the current cell ($\HeadValue$).
	\item an operation: reset ($=0$), add to ($+=$) or subtract from ($-=$).
	In case of reset, nothing else is needed.
	\item a positive multiplicative constant ($ConstantMul *$)
	\item an operand: the accumulator ($\AccValue$), the current cell ($\HeadValue$) or the unit ($\Unit$).
	%\emph{remark}: Add 1 as an operand for constant. I don't do it cause $\Tape_0$ can be used to hold 1.
\end{itemize}

The computation goes as follows: the tape cells and the accumulators are initially set according to an input, defaulting to zero if unspecified.
Likewise, the head is on the origin, $0$, unless specified otherwise.
The choice of the starting state may also be given by the input, defaulting to the one state marked as the starting state.
%Likewise the stack is initially empty unless specified otherwise.

We walk through the graph, starting at the starting state.

At a computation state, we perform the computation and change the value of the recipient accordingly.
At a moving state, we move the head on the tape according to the direction given.
In both cases above, we then proceed with the next state, $\NextNode$.
%
%At a piling node, $\NextNodePile$ is added to the stack and we proceed to the next node, $\NextNode$.

At a branching state, we go to $\NextNodeMinus$, $\NextNodeEqual$ or $\NextNodePlus$ depending on whether $\AccValue$ is negative, zero or positive respectively.

After visiting a terminal state%
%, we pop the last element of the stack and continue from there.
%If the stack is empty,
the computation ends and the value of the tape, accumulator, and the terminal state we ended on can be used to be interpreted as a result.

\TODO{graph examples}

\subsection{code}

This subsection details a practical language for users to create machines described in the previous section.

This language is denoted by the extension \extensionRLTM.
It ought to enable three things:

\begin{itemize}
	\item defining a machine.
	\item specifying an input, through the tape, accumulator value, etc.
	\item running the machine on a given input
\end{itemize}

In addition, it will be able to be exported into an agc file that simulate the machine or encodes an input.

%\begin{minted}{console}
%python3 <path_py/>lbss.py <path_lbss/>test.lbss <path_agc/>
%\end{minted}
%
%\TODO{how it is done and examples}

While the theoretical model allows to manipulate real numbers, for a code, we can only write the numbers we can write.
For practicality, I choose to limit myself to rational numbers.
They are written numerator slash positive denominator or just as integers if possible.
%Technically, they are parsed as Fraction as in python3 fractions modules, which does give more leeway.
%However, it is done in two different ways depending on whether on machine or initial configuration, so I'm not too comfortable putting it that way; maybe something to work on -on the .py as well.
For example, "\mintinline{C}{22/7}"
%\begin{minted}{C}
%22/7
%\end{minted}

Rational numbers might not always be the most efficient thing to manipulate, but they do enable exact computation.

\subsubsection{Shell Command}
To run an lbss file, enter the following command.
\begin{minted}{shell}
python3 <lbss.py_path>/lbss.py my_lbss_file.lbss <agc_files_path>
\end{minted}
In that command \mintinline{shell}{<lbss.py_path>} is the location of the \mintinline{shell}{lbss.py} file, the main file of the lbss compiler/interpreter (relative to where the command is executed).
\mintinline{shell}{<agc_files_path>} is the path where the generated agc files ought to go (relative to where the command is executed).
It is optional, and when absent, the generated agc files, if any, will be put in the same folder as your lbss files.

\subsubsection{Initializing the tape and the machine}
To create a machine, the keyword \functionCreateRLTMmachine is used, followed by two less-than signs and the name of the machine to create.
The definition of the machine follows and a line containing nothing but the name of the machine denotes the end of  that definition.


To create an initial configuration, the keyword \functionCreateRLTMconfiguration is used, followed by two less-than signs and the name of the configuration to create.
The definition of the configuration follows and a line containing nothing but the name of the configuration denotes the end of  that definition.
The definition is done in python, in an environment where \mintinline{python}{x} is the tape, a bi-infinite list whose cells default to 0.

\begin{minted}[linenos]{C}
create_lbss_configuration << example_configuration
x[-1] = Fraction(-4, 3)
x[0] = '3/5'
x[1] = -7
example_configuration
\end{minted}
The values are interpreted as fraction, as taken from the fractions module, hence the quotes around \mintinline{C}{'3/5'}, without them, conversion to float would mess things up.
\TODO{nice info to have around, probably not as detailed for an article, so just a reminder to abridge when I get to it}

\subsubsection{coding a machine}

To create a machine, the keyword \functionCreateRLTMmachine followed by two less-than signs and the name of the machine to create is used.
The definition of the machine follows and a line containing nothing but the name of the machine denotes the end of  that definition.

For example:
\begin{minted}{C}
create_lbss_machine << simplistic_example
	end;
simplistic_example
\end{minted}

Creates a machine named "simplistic\_example", whose definition spans over a single line, that I'm about to explain.
Each state roughly corresponds to one line of code.
Precisely, a semi colon '\mintinline{C}{;}' denotes the end of the definition of a single state (and nothing else).
A terminal state is written simply:
\begin{minted}{C}
	end;
\end{minted}


Left and right states are denoted with lesser-than and greater-than symbols respectively:
\begin{minted}{C}
	>;
	<;
	end;
\end{minted}
is the definition of a machine that goes one right, one left, and stops.


The accumulator and the tape cell where the head is currently at are denoted \mintinline{C}{a} and \mintinline{C}{h} respectively, 
whereas \mintinline{C}{u} stands for unit, and is worth 1.
The reset, add and subtract operations are denoted \mintinline{C}{=0} (or \mintinline{C}{= 0}), \mintinline{C}{+=} and \mintinline{C}{-=} respectively.
A computation state is simply encoded by the succession of recipient, operation, and when relevant, a constant factor, a star '\mintinline{C}{*}' (denoting the product) and an operand.
The constant is positive and mandatory, even if it \mintinline{C}{1}.
For example:
\begin{minted}[linenos]{C}
a = 0;
a += 1 * h;
a += 2 * a;
a += 1/2 * u;
end;
\end{minted}
Sets the value of the accumulator to $3 \HeadValue + 1/2$ - in order, it sets it to $\HeadValue$, adds twice itself to itself then adds half a unit.

See also swap.lbss

Labels are used for branching.
A label is defined by writing a sequence of alphanumerical characters followed by a colon '\mintinline{C}{:}', and points at the next defined state.
The following example denotes a label
\begin{minted}{C}
	LabelX:
\end{minted}

See also Euclide.lbss

Branch states are denoted "\mintinline{C}{Branch}", followed by three comma separated label, corresponding to the three next states $\NextNodeMinus$, $\NextNodeEqual$ or $\NextNodePlus$.
Example:
\begin{minted}[linenos]{C}
a = 0;
a += h;
Branch LabelNegative, LabelZero, LabelPositive;
LabelNegative:
end;
LabelZero:
end;
LabelPositive:
end;
\end{minted}
The above code defines a machine that ends at different states depending on the sign of the value hold in the cell it initially points at.

By default the next state of a computation or move state is the next defined state.
Labels are also used, along with gotos allow to alter that behavior.
Example:

\begin{minted}{C}
	h =0; goto labelEnd
\end{minted}
\TODO{add a more fledge example}

See also slanted.lbss

\subsubsection{Running a machine}
Running a machine is done via the function \mintinline{C}{run}
This function has 3 mandatory arguments: the machine, the input tape, and the name of the run.
\begin{minted}{C}
run machine_swap configuration_swap run_swap
\end{minted}

Additionally, it has 3 optional arguments, the maximum number of steps, the initial node and the initial value of the accumulator, that are to be added in order.
\begin{minted}{C}
run machine_move configuration_move run_move -1 _#0 45
\end{minted}
The default value for the maximum number of steps is $-1$, and signifies infinity.

The initial node can be given as a label, or as a \emph{node identifier} \mintinline{C}{labelName_<number>}; it defaults  to \mintinline{C}{_#0}  which is the identifier of the first node of the code, provided it does not start with a label.

The default value for the accumulator is $0$.

\subsubsection{Running a unary binary tree drawing}
The following function is used to draw a unary binary tree from an lbss machine, according to the Delay and Split nodes it stops by.
\begin{minted}{C}
draw_ubt machine_name configuration_name 10 initialNode
\end{minted}
The machine and configuration are mandatory, other arguments are optional, to be given in order.
The third argument is the maximum depth, which defaults to 7.
The last argument is the initial node, which again can be given as a label or node identifier ad defaults to \mintinline{C}{_#0}.


\subsubsection{Exporting to agc}
The keywords \functionToAGCmachine, \functionToAGCconfig and \functionToAGCrun are used to generate .agc files of the corresponding objects.
Typically, machines and configurations generated by \functionToAGCmachine and \functionToAGCconfig respectively will be used by the run file generated by \functionToAGCrun.
That latter file is typically the one you want to run the agc software on, and generates a pdf time space diagram of the machine run.

To compile a machine into an agc file, use:
\begin{minted}{C}
toAGC_machine machine_swap machine_swap.agc
\end{minted}

To compile a configuration, use:
\begin{minted}{C}
toAGC_config configuration_swap configuration_swap.agc 3
\end{minted}
The last argument is the initial value of the accumulator, here $3$.
It is optional and defaults to 0.

To compile a run, use:
\begin{minted}[breaklines]{C}
toAGC_run machine_swap.agc configuration_swap.agc test_machine_swap.agc test_machine_swap.pdf _#0 1/6 1000
\end{minted}
\begin{itemize}
	\item The machine and configuration can either be given as agc filenames, or directly as machine and configuration names.
	Caution though, machine and configuration will still be compiled, without a name of their own.
	Don't cut corners more than once per agc folder.
	\TODO{improve/change compiler behavior and reflect}
	\item The name of the pdf is optional and defaults to \mintinline{C}{"lbssagc_out.pdf"}.
	\item The next optional argument is the initial node, which defaults to \mintinline{C}{_#0}.
	\item The next argument is the scale, which relates to the thickness of the signal's drawing and defaults to $1/3$.
	\item The last arguments is the maximum number of steps to run the agc machine;
	it is the number of signal collision and is only loosely related to the number of steps of the lbss machine simulated (one lbss steps is simulated with a couple tens agc steps, give or take).
\end{itemize}

The result is a file that can be run with agc software to generate a pdf.
In the previous example, the command to run the agc would be:
\begin{minted}{shell}
java -jar agc_2_0.jar <agc_path>/test_machine_swap.agc
\end{minted}
It's important to note that agc files ought to be run from the same place they were generated.
%or as the lbss files used to generate them were? TODO check.
%The reason is essentially to avoid path substraction. The AGC files need to point at the agc library, whose path is inferred from the .py files path, upon running the lbss files.

\section{Technical part about AGC implementation}

\TODO {define simulate}

An lbss\_agc machine is a machine that simulates an \RLTM machine.
What simulate means is detailed in the first subsection.

\subsection{primitives in agc}
The parts of the machines and initial configurations that are common to all lbss\_agc machines.

\subsubsection{The tape and the accumulator}

As shown in \RefFig{fig:initialConfiguration:singleRegister}, a value is encoded as the (oriented) distance of a value signal from a zero signal.
The zero signal is pain blue, while the value signal is densely dashed.
The value is bounded by two signals, one red and one green, for overflow detection.
For now, nothing is done if the value gets too big.
Such a complex of up to 4 signals is called a \emph{register}.
\begin{figure}[hbt]
	\centering
	\small%
	\SetUnitlength{.14\textwidth}%
	\includegraphics[height=\unitlength]{initialConf_singleCell_noAccu.pdf}%
	\caption{A single cell.}
	\label{fig:initialConfiguration:singleRegister}
\end{figure}

\TODO{CR and MS definition tables}

Each cell has a black marker, at the left of its value, to help the head movement.
Each cell has room to host the computing head, which comprises a program signal, a constant register and the accumulator.

\RefFigure{fig:initialConfiguration:singleCellWithAccu} shows a cell with the head pointed on it, but without program signal.
The rightmost register stores the cell value, as previously.
The middle register is the accumulator.
The leftmost register lacking a left bound holds the constant value 1.

\begin{figure}[hbt]
	\centering
	\small%
	\SetUnitlength{.14\textwidth}%
	\includegraphics[height=\unitlength]{initialConf_singleCell.pdf}%
	\caption{A cell where the head points.}
	\label{fig:initialConfiguration:singleCellWithAccu}
\end{figure}


\RefFigure{fig:initialConfiguration:tape} shows a tape with 3 cells.
\RefFigure{fig:initialConfiguration:tape:noAccu} shows it without head, whereas \RefFigure{fig:initialConfiguration:tape:withAccu} shows it with a head pointing on the middle cell.

\FRIDGE{If a number gets too big, they are all scaled down.
	With a spacing between cell proportional to the biggest multiplicative constant of the program and to the space between bounds, one can ensure that an operation on bounded values does not spill on other values.
	
	Unlike tape and accumulator initialization, initial configuration also depend on the machine.
	No big deal because 
	firstly it could be made independent (scaling down stuffs once at the beginning) and 
	secondly, it is not the main obstacle to universality (see section \ref{sec:universality}).
}

\begin{figure}[hbt]
	\centering
	\small%
	\SetUnitlength{.05\textwidth}%
	\subcaptionbox{no head%
	\label{fig:initialConfiguration:tape:noAccu}}{%
	\includegraphics[height=\unitlength]{initialConf_3cells_noAccu.pdf}%
	}
	\SetUnitlength{.045\textwidth}%
	\subcaptionbox{a head points on the middle cell%
	\label{fig:initialConfiguration:tape:withAccu}}{%
	\includegraphics[height=\unitlength]{initialConf_tape.pdf}%
	}
	\caption{A tape with 3 cells.}
	\label{fig:initialConfiguration:tape}
\end{figure}

For now the tape is finite.
\FRIDGE{
The tape is finite and expansion to simulate an infinite tape is handled by the moving head.
}

\subsubsection{the Head}

The head is composed of a program signal, then (to its right) the constant register and the accumulator.
Further right is the cell value, which isn't part of the head per se, but can obviously be interacted with.
Unless specified otherwise, there is one meta-signal per machine state for the program signal to keep track of the state.

During the execution of the program, a signal will go back and forth between the data (constant, accumulator and current cell) and the program signal.

Going from the program signal to the data, that signal is called a go signal.
%It can only be of one meta-signal, $\goMS$.

Going from the data to the program signal, that signal is called a return signal.
It can be of several type, most notably when returning from a branch test.

\TODO{reference subsection}
For the rest of this subsection, we ignore the program signal part and focus on what primitive operations happen between two program states.
Each primitive will thus be started by a go signal (right after it interacted with the program) and will end with a return signal (right before it interact with the program)

%||, >>, <<, <<+, <<=, <<-, <<0, prgX

\subsubsection{the moving head}
Moving is handled by parallelograms and using the markers, as shown by \RefFig{fig:move:right}

\begin{figure}[hbt]
	\centering
	\small%
	\SetUnitlength{.17\textwidth}%
	\includegraphics[height=\unitlength]{move_right.pdf}%
	\caption{Moving the head one cell to the right.}
	\label{fig:move:right}
\end{figure}

\FRIDGE{
At each end of the tape, the last cell has a special marker and there is an other special marker where the empty markers of the next cell would be.
They are used to create a new cell upon reaching the last one.
}

\subsubsection{reset}
Resetting is done simply with a go a return signal that cuts the value signal to reset and marks the zero signal as value, as shown in \RefFig{fig:reset}.
\begin{figure}[hbt]
	\centering
	\small%
	\SetUnitlength{.25\textwidth}%
	\includegraphics[height=\unitlength]{reset_cell.pdf}%
	\caption{Resetting the cell value to 0.}
	\label{fig:reset}
\end{figure}

\subsubsection{reading data}
Add and subtract states are cut in two elementary phase, corresponding to three program meta-signals.

The first phase, called get, reads a positive constant times a register and and holds the value in a temporary form called \emph{temporary register}. This temporary register consists in 2 parallel signals of different type, one corresponding to "zero", the other to "value". The value is encoded by the vertical space between them. The sign is encoded by their relative position, zero above value being positive. The sole purpose of this register is to be added or subtracted to a register according to the basic lbss command that prompted the get.

For example, in:
\begin{minted}{C}
a += 1/2 * h;
\end{minted}
if we refer to the temporal register via the variable '\mintinline{C}{t}' the get action corresponds to
\begin{minted}{C}
t <- 1/2 * h;
\end{minted}
and the second phase corresponds to
\begin{minted}{C}
a <- a + t;
\end{minted}

In \RefFigure{fig:get:accu} shows how a register value (here the accumulator's) is read into the temporary register. The difference between \RefFigure{fig:get:accu:1} and \RefFigure{fig:get:accu:7:2} shows that by changing some speeds, a value can be multiplied before being read.

\begin{figure}[hbt]
	\centering
	\small%
	\SetUnitlength{.15\textwidth}%
	\subcaptionbox{getting the accumulator value%
		\label{fig:get:accu:1}}{%
		\includegraphics[height=\unitlength]{get_accu_times1.pdf}%
	}
	\SetUnitlength{.21\textwidth}%
	\subcaptionbox{getting the accumulator value multiplied by $3.5$%
		\label{fig:get:accu:7:2}}{%
		\includegraphics[height=\unitlength]{get_accu_times7:2.pdf}%
	}
	\caption{Get, with different multiplicative constant.}
	\label{fig:get:accu}
\end{figure}


\subsubsection{Addition and Subtraction}
\TODO{more readable.}
The transition from get to addition/subtraction is later covered in subsection \ref{subsec:Compiling}.
It's still worth mentioning here that we assume the temporary register holds a positive value, and switch operations if necessary.

Assuming the temporary register holds a value to add to the accumulator, it is done as shown by \RefFig{fig:add:accu:45:4}
\begin{figure}[hbt]
	\centering
	\small%
	\SetUnitlength{.17\textwidth}%
	\includegraphics[height=\unitlength]{add_accu45:4.pdf}%
	\caption{Adding 45/4 to the accumulator.}
	\label{fig:add:accu:45:4}
\end{figure}

Assuming the temporary register holds a value to subtract from the cell, it is done as shown by \RefFig{fig:add:sub:cell}
\begin{figure}[hbt]
	\centering
	\small%
	\SetUnitlength{.25\textwidth}%
	\includegraphics[height=\unitlength]{sub_cell.pdf}%
	\caption{subtracting from the cell.}
	\label{fig:sub:cell}
\end{figure}

\subsubsection{branch}

Branching is done as in \RefFigure{fig:branch:neg}: a signal goes and probes whether the value, the zero, or both (as a superposition) is encountered first.
The type of the return signal thus depends on the sign of the accumulator, and is used as later shown in subsection \ref{subsec:Compiling}.
\TODO{define MS and use them}
\begin{figure}[hbt]
	\centering
	\small%
	\SetUnitlength{.14\textwidth}%
	\includegraphics[height=\unitlength]{branch_negative.pdf}%
	\caption{Branching.}
	\label{fig:branch:neg}
\end{figure}


\FRIDGE{
\subsubsection{FRIDGE boundaries}
If the tape becomes too big, scale everything down.

If the computation takes too much time, scale everything down.

Both in a way to have remain under a known time constraint (it's not enough that it's finite, there needs to be a bound and we need to know it).
}

\subsection{Compiling RLTMmachine into agc\RLTM machine}
\label{subsec:Compiling}
The meta-signal ids of the states are 
$L\#n$
where $L$ is the last label if any, and $n$ is the number of states since that last label, or since the beginning of the code (everything happen as if there is an empty string label at the beginning of each code).
These meta-signals have speed $0$.
Even though these IDs are unique, faulty but sensible ID referencing is corrected on the fly, producing a warning, and allowing dubious practice such as:
\begin{itemize}
	\item Double labeling.
	\item Labeling the beginning of the code and not explicitly giving the starting label.
	\item omitting terminal nodes.
\end{itemize}

Once a non computation state is read (up to the semi colon), we know its id and its nature.
The upper half of a program collision is then set to the program meta-signal itself, and the corresponding instruction meta-signal.

Then reading the label(s) explicitly or implicitly referred to (goto, next line, branch...), we can infer the corresponding ids of the next state(s).
The program collision is then set as follows: the lower half is the program meta-signal of the instruction that were just ran and a return meta-signal, the upper half is those of whichever state comes next, if any, which may or may not depend on the return signal.

Computation instructions are treated somewhat differently in two regards.
Firstly, syntax such as 
\begin{minted}{C}
a = -4;
\end{minted}
is allowed, even though it does not corresponds to a computation state as described in subsection \ref{subsec:RLTMdef}.
The first step is thus to break down \RLTM instruction into basic RLTM states.
In our example, that would be a state $a = 0$ followed by a state $a -= 4 * u$.

Secondly, non reset computation state are broken into three consecutive program meta-signals:
\begin{itemize}
	\item a get meta-signal.
	\item an auxiliary get meta-signal, which prompts the operation add or sub according to the signs of the constant, the sign value read with the first return signal and the operation desired.
	\item a final meta-signal which bounces the upper part of the temporary register, and links to the next machine state.
\end{itemize}

Other than being three, these program meta-signals are treated like non computation ones, as described earlier.
Some suffixes are used for their IDs so as to keep the correspondence between number of states and ID number, the $n$ in $L\#n$.
There can be discrepancies between that number of states and the number of instructions (ending with a semi colon) due to how some computation nodes can be written as a single \RLTM instruction but are broken into several states.


\subsection{inputs into configuration}
Straightforward.

The computation is started by having the upper part of a program collision corresponding to the starting state.

\section{Unary binary tree}

The idea is to build a unary binary tree, with an LBSS machine providing the orders, by stopping on a given terminal node.

\subsection{Primitives}
\subsubsection{Delay}
\TODO{txt for illustration}
Doing nothing during a given amount of time carefully measured with auxiliary bouncing signal.

\begin{figure}[hbt]
	\centering
	\small%
	\SetUnitlength{.14\textwidth}%
	\includegraphics[height=\unitlength]{ubt_upright_to_Delay_.pdf}%
	\caption{Delay from a upward right branch.}
	\label{fig:delay:upright}
\end{figure}

\begin{figure}[hbt]
	\centering
	\small%
	\SetUnitlength{.14\textwidth}%
	\includegraphics[height=\unitlength]{ubt_right_to_Delay_.pdf}%
	\caption{Delay from a rightward branch.}
	\label{fig:delay:right}
\end{figure}


\subsubsection{split}
\TODO{txt for illustration}
The information is duplicated and split into a right and a left branch.

\begin{figure}[hbt]
	\centering
	\small%
	\SetUnitlength{.14\textwidth}%
	\includegraphics[height=\unitlength]{ubt_right_to_Split_.pdf}%
	\caption{Split from a rightward branch.}
	\label{fig:split:right}
\end{figure}

\begin{figure}[hbt]
\centering
\small%
\SetUnitlength{.14\textwidth}%
\includegraphics[height=\unitlength]{ubt_upleft_to_Split_.pdf}%
\caption{Splitting from a upward left branch.}
\label{fig:split:upleft}
\end{figure}


\subsection{Wiring with LBSS}

There is one LBSS machine coding a class of parameterized functions.
The initial configuration corresponds to the parameters.
Two new kinds of states are added:

\begin{itemize}
	\item \Delay: signals the next tree node has a single child and suspends computation until the next tree node.
	It has exactly one next state, indicating where to resume computation.
	\item \Split: signals the next tree node has two children (and it halves the scale) and suspends computation until the next tree node.
	It has exactly two next states, indicating where to resume computation on each children.
	In the code, it translates to 2 mandatory labels, corresponding to left and right, in that order.
\end{itemize}

Example code for a delay:
\begin{minted}{C}
	Delay;
\end{minted}

Example code for a delay with a goto:
\begin{minted}{C}
	Delay; goto labelFromDelay
\end{minted}

Example code for a split:
\begin{minted}{C}
	Split labelFromSplitToLeft, labelFromSplitToRight
\end{minted}

Example code for slanted:
\begin{minted}[breaklines, linenos]{C}
left:
Split left, right;
right:
Delay left;
end;
machine_ubtTest_rec
\end{minted}
This machine will recursively draw a tree where leftward and upward branches are followed by a split and right branches are followed by a delay.

\subsection{AGC Implementation}
\TODO{fluff up (mscr) and tidy up, cite sfss}
The tree is organized with Macro Signals: a band of signals bounded by the tree signal, and a bounding signal.

The tree steps are performed with the help of a pair bouncing signals
They correspond to both bounds of the macro signal.
The first intersection with bouncing signals starts the macro collision and the second marks the exact position of the Tree Node.

The information about whether the next node is $\Delay$ or $\Split$ is on the tree in an up branch, on the bound otherwise.


\begin{figure}[hbt]
	\centering
	\small%
	\SetUnitlength{1.0\textwidth}%
	\includegraphics[height=\unitlength]{ubt_agc.pdf}%
	\caption{Delay from a rightward branch.}
	\label{fig:ubtagc:example1}
\end{figure}

\subsubsection{Stopping Condition}
For now, it is assumed that the number of steps by the lbss machine is bounded, with a known upper bound.
\FRIDGE{(fridge) We assume it always end and we can compute it in finite time.
If it cannot, it fails (the simulation does not go further).}

If the machine ends on a split or a delay, the information is reported where  it should for the next node.
If it ends on an ending node, no information is reported and it prompts the stopping of this node.
Additionally -and optionally, a maximum depth, tracked by the tree meta-signals, is also used to stop the drawing of the tree.
It works by simply not prompting the computation once the maximum depth is reached, so that no result is reported.

\TODO{rerouting}
\TODO{lbss incrustration}

\section {universality}
\label{sec:universality}
Techniques, ad hoc or general (Iran paper) can be employed to make more general agc machines capable of simulating whole classes of lbss\_agc machines each.

One can only do so much though, a priori, there is no universal linear Turing machine, the most one can do is universal for a set of multiplicative constants, much like agc machine can only be universal for a set of meta-signal speeds.
Emphasis on "a priori" since simulating is a fuzzy concept arbitrarily defined and I haven't dug deep into it.

%TODO rewrite, move or remove.
%\section{Technical Compiler and Interpreter part (Python)}
%
%\emph{Language:}
%Python.
%
%Simple to use, I don't really need more.
%
%Other candidates were Ruby and Lua, for rational expression.
%Ruby I don't know.
%Lua I love, but for this purpose, the difference with python is lesser popularity and no structure other than tables.
%
%String processing library-ies?
%RE
%
%\subsection{main program}
%Goal: identify what file has to be loaded, what string has to be read...
%Then what to interpret, what to compile and how.
%
%Presumably, it will be pattern search (<<< should allow to avoid good parenthesis, but is not an issue if needed).
%
%The lists of whats to identify:
%\begin{itemize}
%	\item machine definition (file or string). A python object (dico?) has to be built.
%	\item configuration definition (file or string). A python object (dico) has to be built.
%	\item interpreting a machine. Has to be done in python (feed relevant dico to a function)
%	\item compiling something. python object to feed to a function that writes in an agc file
%\end{itemize}
%
%\subsection{initial configuration loader}
%The string is interpreted as python code, and ought to define either a list (the tape)
%or a dictionary with a field "tape" (a list of numbers), a field "accumulator" (the initial accumulator value) and a field "initial\_state" (a string), all optional.
%
%The loader transforms it into a dic with all fields filled, to be used by the interpreter or the loader.
%
%\subsection{interpreter}
%graphic library or full text (console/ log file)?
%Full text.




%%%%%%%%%%%%%%%%%%%%%%%%%%%%%%%%%%%%%%%%%%%%%%%%%%%%%%%%%%%%%%%%%%%%%%%%%%%%%%%%%%%%%%%%%% 
\small
 \bibliographystyle{alpha}
% \bibliographystyle{plainnat-letters}
% \bibliographystyle{plainnat}%unsrtnat
% \bibliographystyle{abbrvnat}%unsrtnat
%\bibliographystyle{splncs03}%unsrtnat


\bibliography{bib_jdl,bib_paper}

%%%%%%%%%%%%%%%%%%%%%%%%%%%%%%%%%%%%%%%%%%%%%%%%%%%%%%%%%%%%%%%%%%%%%%%%%%%%%%%%%%%%%%%%%% 

\clearpage

\appendix


\end{document}